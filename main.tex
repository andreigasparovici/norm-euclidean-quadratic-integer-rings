\documentclass[12pt]{article}
\usepackage{geometry}
\usepackage{amsmath}
\usepackage{amsfonts}
\usepackage{hyperref}

\hypersetup{
    colorlinks=true,
    linkcolor=blue,
}

\geometry{a4paper, total={6.5in, 10in}}

\pagenumbering{gobble}

\title{Inele de întregi pătratici euclidiene în raport cu funcţia normă}
\author{Andrei Gasparovici}
\date{6 Noiembrie 2020}

\begin{document}

\maketitle

Fie $m, n \in \mathbb{Z}$ şi $\theta, \theta' \notin \mathbb{Q}$ rădăcinile ecuaţiei

\begin{equation} \label{eq:def_theta}
    x^2 + mx + n = 0, \; m, n \in \mathbb{Z}
\end{equation}

şi fie un inel de întregi pătratici %\textbf{euclidian}

\begin{equation} \label{eq:def_z_theta}
    \mathbb{Z}[\theta] = \left\{ a + \theta b \,|\, a, b \in \mathbb{Z}\right\}
\end{equation}

\vspace{.8cm}
Ne propunem să găsim $\theta \in \mathbb{C} \backslash \mathbb{Q}$ pentru care $\mathbb{Z}[\theta]$ este \textbf{euclidian}.
\vspace{.8cm}

Definim pe $\mathbb{Z[\theta]}$ funcţia normă

\begin{equation} \label{eq:def_norm}
    N : \mathbb{Z[\theta]} \to \mathbb{N}, \, N(a + \theta b) = |(a + \theta b)(a + \theta' b)|
\end{equation}

cu proprietăţile cunoscute:

\begin{equation*}
N(z) = 0 \Leftrightarrow z = 0, \quad \forall z \in \mathbb{Z}[\theta]
\end{equation*}
\begin{equation*}
N(z_1 z_2) = N(z_1)N(z_2), \quad \forall z_1, z_2 \in \mathbb{Z}[\theta]
\end{equation*}

Fie $z_1, z_2 \in \mathbb{Z}[\theta]$, $z_2 \neq 0$. Arătăm că:

\begin{equation*}
    \frac{z_1}{z_2} = r + \theta s, \quad r, s \in \mathbb{Q}è
\end{equation*}

Fie $z_1 = a_1 + \theta b_1$, $z_2 = a_2 + \theta b_2, a_1,a_2,b_1,b_2 \in \mathbb{Z}$. Atunci:

\begin{equation} \label{eq:ratio}
    \frac{z_1}{z_2} = \frac{a_1 + \theta b_1}{a_2 + \theta b_2} = 
    \frac{(a_1 + \theta b_1)(a_2 + \theta' b_2)}{(a_2 + \theta b_2)(a_2 + \theta' b_2)}
\end{equation}


Calculăm numitorul din raportul \eqref{eq:ratio}:

\begin{equation} \label{eq:ratio_denominator}
    (a_2 + \theta b_2)(a_2 + \theta' b_2) = a_2^2 + a_2b_2(\theta + \theta') + b_2^2\theta\theta'
\end{equation}

Cu relaţiile lui Viète în \eqref{eq:def_theta} avem:

\begin{equation} \label{eq:viete}
    %\begin{array}{ll}
    \begin{cases}
    \theta + \theta' &= -m \\
    \theta \cdot \theta' &= n \\
    %\end{array}
    \end{cases}
\end{equation}

Relaţia \eqref{eq:ratio_denominator} devine:
\begin{equation}
    (a_2 + \theta b_2)(a_2 + \theta' b_2) = a_2^2 - m a_2 b_2 + n b_2^2 \in \mathbb{Z}
\end{equation}

Pentru cazul $a_2^2 - m a_2 b_2 + n b_2^2 > 0$, raportul \eqref{eq:ratio} devine:

\begin{equation} \label{eq:ratio_a}
    \frac{z_1}{z_2} = \frac{(a_1 + \theta b_1)(a_2 + \theta' b_2)}{(a_2 + \theta b_2)(a_2 + \theta' b_2)} 
    =  \frac{(a_1 + \theta b_1)(a_2 + \theta' b_2)}{|(a_2 + \theta b_2)(a_2 + \theta' b_2)|} 
    = \frac{a_1 a_2 + \theta' a_1 b_2 + \theta a_2 b_1 + \theta \theta' b_1b_2}{N(z_2)}
\end{equation}

Aplicând \eqref{eq:viete}, obţinem că

\begin{equation} \label{eq:ratio_a_viete}
    \frac{z_1}{z_2} =
    \frac{a_1 a_2 - m a_1 b_2 + \theta a_2 b_1 + n b_1b_2}{N(z_2)} =
    \frac{q_1 + \theta q_2}{N(z_2)} = 
    r + \theta s, \quad r, s \in \mathbb{Q}
\end{equation}

Analog în cazul $a_2^2 - m a_2 b_2 + n b_2^2 < 0$.

\vspace{.4cm}
Definim

\begin{equation} \label{eq:def_q_theta}
    \mathbb{Q}[\theta] = \left\{ a + \theta b \,|\, a, b \in \mathbb{Q} \right\}
\end{equation}

şi funcţia

\begin{equation} \label{eq:def_phi}
    \varphi : \mathbb{Q}[\theta] \to \mathbb{N}, \, \varphi(a + \theta b) = |(a + \theta b)(a + \theta' b)|
\end{equation}

Se observă că restricţionând funcţia $\varphi$ la $\mathbb{Z}[\theta]$ obţinem funcţia normă:

\begin{equation} \label{eq:phi_restricted_z}
    \left.\varphi\right|_{\mathbb{Z}[\theta]} = N
\end{equation}

Inelul $\mathbb{Z}[\theta]$ este \textbf{euclidian} în raport cu funcţia normă. Deci:

\begin{equation} \label{eq:z_theta_euclidean}
    \begin{split}
	\forall z_1, z_2 \in \mathbb{Z}[\theta] \quad \exists \, c, r \in \mathbb{Z}[\theta] \quad \text{a.î.} \quad
        & z_1 = z_2 c + r, \text{ cu } r = 0 \text{ sau } N(r) < N(z_2) \\
        &\Leftrightarrow N(z_1 - z_2 c) < N(z_2) \\
        &\Leftrightarrow \varphi \left(\frac{z_1}{z_2} - c\right) < 1
    \end{split}
\end{equation}

Fie $c = x + \theta y$, $x, y \in \mathbb{Z}$.
Atunci conform \eqref{eq:viete}, \eqref{eq:ratio_a_viete} şi \eqref{eq:def_phi}:

\begin{equation} \label{eq:phi_condition}
\begin{split}
    \varphi  \left(\frac{z_1}{z_2} - c\right) & =\varphi  \left(\frac{z_1}{z_2} - c\right) \\
    &=\varphi(r + \theta s - x - \theta y) \\
    &= \varphi \left( (r-x) + \theta(s-y) \right) \\
    &=  \left| (r-x)^2 - m(r-x)(s-y) + n(s-y)^2\right|
\end{split}
\end{equation}

În continuare, vom studia două cazuri particulare ale ecuaţiei \eqref{eq:def_theta}.

\section*{Cazul 1} 
\begin{equation} \label{eq:case_1}
    x^2 - d = 0, \, d \in \mathbb{Z} \Rightarrow \begin{cases} m = 0 \\ n = -d \end{cases}
\end{equation}

Relaţia \eqref{eq:phi_condition} devine:

\begin{equation} \label{eq:phi_condition_case1}
    \varphi \left(\frac{z_1}{z_2} - c\right)  = \left| (r-x)^2 - d(s-y)^2 \right| 
   \Rightarrow \varphi \left(\frac{z_1}{z_2} - c\right) \leq (r-x)^2 + |d|(s-y)^2
\end{equation}

Alegem

\begin{equation*}
\begin{split}
    x = \left[ r + \frac{1}{2} \right] \in \mathbb{Z} \Rightarrow & x \leq r + \frac{1}{2} < x + 1 \\
    & -\frac{1}{2} \leq r - x < \frac{1}{2} \\
    & \boxed{|r - x| \leq \frac{1}{2}}
\end{split}
\end{equation*}

şi

\begin{equation*}
\begin{split}
    y = \left[ s + \frac{1}{2} \right] \in \mathbb{Z} \Rightarrow \boxed{|s - y| \leq \frac{1}{2}}
\end{split}
\end{equation*}

Inecuaţia din \eqref{eq:phi_condition_case1} devine:

\begin{equation*}
    \varphi \left(\frac{z_1}{z_2} - c\right) \leq \frac{1}{4}(1 + |d|)
\end{equation*}

Aplicând \eqref{eq:z_theta_euclidean} rezultă că:

\begin{equation*}
    1 + |d| < 4 \Leftrightarrow |d| < 3
\end{equation*}

Cum $d \in \mathbb{Z}$ \eqref{eq:case_1}, obţinem:

\begin{equation*}
    d \in \{\pm 1, \pm 2\}
\end{equation*}

Pentru $d = 1$:

\begin{equation*}
    x^2 - 1 = 0 \Rightarrow \theta, \theta' = \pm 1 \in \mathbb{Q}
\end{equation*}

Pentru $d = -1$:

\begin{equation*}
    x^2 + 1 = 0 \Rightarrow \theta, \theta' = \pm i \notin \mathbb{Q} \Rightarrow \boxed{\mathbb{Z}[i] \text{ euclidian}}
\end{equation*}

Pentru $d = 2$:

\begin{equation*}
    x^2 - 2 = 0 \Rightarrow \theta, \theta' = \pm \sqrt{2} \notin \mathbb{Q} \Rightarrow \boxed{\mathbb{Z}[\sqrt{2}] \text{ euclidian}}
\end{equation*}

Pentru $d = -2$:

\begin{equation*}
    x^2 + 2 = 0 \Rightarrow \theta, \theta' = \pm i\sqrt{2} \notin \mathbb{Q} \Rightarrow \boxed{\mathbb{Z}[i\sqrt{2}] \text{ euclidian}}
\end{equation*}

\section*{Cazul 2} 
\begin{equation} \label{eq:case_2}
x^2 + x + \frac{1 - d}{4} = 0, \, d \in \mathbb{Z},
\text{ cu } d \equiv 1 \; (\text{mod}\ 4), \; d \text{ liber de pătrate }
\Rightarrow \begin{cases} m = 1 \\ n = \frac{1-d}{4} \end{cases}
\end{equation}

Relaţia \eqref{eq:phi_condition} devine:

\begin{equation} \label{eq:phi_condition_case2}
\begin{split}
    \varphi \left(\frac{z_1}{z_2} - c\right)  &= \left| (r-x)^2 - (r-x)(s-y) + \frac{1-d}{4}(s-y)^2 \right| \\
    &= \left| \left(r + \frac{s}{2} - x - \frac{1}{2}y\right)^2 - \frac{d}{4}(s-y)^2 \right|
\end{split}
\end{equation}

Notând

\begin{equation*}
\begin{split}
    & r' = r + \frac{s}{2} \\
    & s' = \frac{s}{2}
\end{split}
\end{equation*}

Relaţia \eqref{eq:phi_condition_case2} devine:

\begin{equation} \label{eq:phi_condition_case2_subst}
    \varphi \left(\frac{z_1}{z_2} - c\right) =
    \left| \left(r' - x - \frac{1}{2}y \right) ^2  + |d \left(s' - \frac{1}{2}y\right)^2  \right|
\end{equation}

Alegem

\begin{equation*}
\begin{split}
    y = \left[ 2s' + \frac{1}{2} \right] \in \mathbb{Z} \Rightarrow & 
    y \leq 2s' + \frac{1}{2} < y + 1 \\
    & -\frac{1}{2} \leq 2s'- y < \frac{1}{2} \\ 
    & \left| 2s' - y \right| \leq \frac{1}{2} \\
    & \boxed{\left| s' - \frac{y}{2} \right| \leq \frac{1}{4}}
\end{split}
\end{equation*}

şi

\begin{equation*}
    x = \left[ r' - \frac{y}{2} + \frac{1}{2} \right] \in \mathbb{Z} \Rightarrow 
    \boxed{|r' - x - \frac{1}{2}y| \leq \frac{1}{2}}
\end{equation*}


Din \eqref{eq:phi_condition_case2_subst} rezultă:

\begin{equation*}
    \varphi \left(\frac{z_1}{z_2} - c\right) \leq
    \left( r'- x - \frac{1}{2}y \right)^2 + |d|\left(s' - \frac{y}{2}\right)^2 \leq \frac{1}{4} + \frac{1}{16}|d|
\end{equation*}

Aplicând \eqref{eq:z_theta_euclidean}, obţinem:

\begin{equation*}
    \frac{1}{4} + \frac{1}{16}|d| < 1 \Leftrightarrow |d| < 12
\end{equation*}

Conform $d \equiv 1 \; (\text{mod}\ 4), \; d \text{ liber de pătrate }$ \eqref{eq:case_2}:

\begin{equation*}
    d \in \{\pm 3, \pm 5, \pm 7, \pm 11\}
\end{equation*}

Pentru $d = 3$:

\begin{equation*}
    x^2 - x + \frac{1 - 3}{4} = 0
    \Leftrightarrow x^2 - x - \frac{1}{2} = 0; \; -\frac{1}{2} \notin \mathbb{Z}
\end{equation*}

Pentru $d = -3$:

\begin{equation*}
    x^2 - x + \frac{1 + 3}{4} = 0
    \Leftrightarrow x^2 - x + 1 = 0
    \Leftrightarrow \theta, \theta' = \frac{1 \pm i\sqrt{3}}{2} \notin \mathbb{Q}
    \Rightarrow \boxed{\mathbb{Z}\left[\frac{1 + i\sqrt{3}}{2}\right] \text{ euclidian}}
\end{equation*}

Pentru $d = 5$:

\begin{equation*}
    x^2 - x + \frac{1 - 5}{4} = 0
    \Leftrightarrow x^2 - x - 1 = 0
    \Leftrightarrow \theta, \theta' = \frac{1 \pm \sqrt{5}}{2} \notin \mathbb{Q}
    \Rightarrow \boxed{\mathbb{Z}\left[\frac{1 + \sqrt{5}}{2}\right] \text{ euclidian}}
\end{equation*}

Pentru $d = -5$:

\begin{equation*}
    x^2 - x + \frac{1 + 5}{4} = 0
    \Leftrightarrow x^2 - x + \frac{3}{2} = 0; \; \frac{3}{2} \notin \mathbb{Z}
\end{equation*}

Pentru $d = 7$:

\begin{equation*}
    x^2 - x + \frac{1 - 7}{4} = 0
    \Leftrightarrow x^2 - x - \frac{3}{2} = 0; \; -\frac{3}{2} \notin \mathbb{Z}
\end{equation*}

Pentru $d = -7$:

\begin{equation*}
    x^2 - x + \frac{1 + 7}{4} = 0
    \Leftrightarrow x^2 - x + 2 = 0
    \Leftrightarrow \theta, \theta' = \frac{1 \pm i\sqrt{7}}{2} \notin \mathbb{Q}
    \Rightarrow \boxed{\mathbb{Z}\left[\frac{1 + i\sqrt{7}}{2}\right] \text{ euclidian}}
\end{equation*}

Pentru $d = 11$:

\begin{equation*}
    x^2 - x + \frac{1 - 11}{4} = 0
    \Leftrightarrow x^2 - x - \frac{5}{2} = 0; \; -\frac{5}{2} \notin \mathbb{Z}
\end{equation*}

Pentru $d = -11$:

\begin{equation*}
    x^2 - x + \frac{1 + 11}{4} = 0
    \Leftrightarrow x^2 - x + 3 = 0
    \Leftrightarrow \theta, \theta' = \frac{1 \pm i\sqrt{11}}{2} \notin \mathbb{Q}
    \Rightarrow \boxed{\mathbb{Z}\left[\frac{1 + i\sqrt{11}}{2}\right] \text{ euclidian}}
\end{equation*}

\section*{Concluzie}
Am găsit că inelele de întregi pătratici
$\mathbb{Z}[i]$,
$\mathbb{Z}[\sqrt{2}]$,
$\mathbb{Z}[i\sqrt{2}]$,
$\mathbb{Z}\left[\frac{1 + i\sqrt{3}}{2}\right]$,
$\mathbb{Z}\left[\frac{1 + \sqrt{5}}{2}\right]$,
$\mathbb{Z}\left[\frac{1 + i\sqrt{7}}{2}\right]$ şi
$\mathbb{Z}\left[\frac{1 + i\sqrt{7}}{2}\right]$
sunt euclidiene în raport cu funcţia normă.

\end{document}
